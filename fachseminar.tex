\documentclass[12pt,a4paper]{article}
\usepackage[utf8]{inputenc}
\usepackage[german]{babel}
\usepackage[T1]{fontenc}
\usepackage{graphicx}

\usepackage[paper=a4paper,width=14cm,left=35mm,height=22cm]{geometry}
\usepackage{setspace}
\linespread{1.15}
\setlength{\parskip}{0.75em}
\setlength{\parindent}{0em}

\bibdata{bibliothek}

\begin{document}
\author{Markus Tacker}
\title{Konzepte und Standards zur domänenübergreifenden Integration von komplexen Webanwendungen}
\maketitle
\section*{Abstract}
Ein Teil des neueren Entwicklung des Internets zum \emph{Web 2.0} basiert auf der Idee, dass Anbieter von Internetdiensten ihre Informationen und Funktionen Hilfe von Schnittstellen Dritten zur Verfügung stellen.\cite{baresitalk2007}

Diese sind auf der Protokollebene standardisiert, müssen jedoch vom Verwender immer individuell verwendeten Dienst implementiert werden, wodurch sich der Verwender fest an einen Dienst bindet.

Für sogenannte \emph{Blackbox-Dienste} ist das kein Problem. Diese Dienste verarbeiten lediglich einfache Daten, d.h. dass der Dienst durch Übergabe eines Datums aufgerufen wird, dieser entsprechende des Aufrufes reagiert und ggfs. ein modifiziertes Datum zurück liefert. Beispiele hierfür sind z.B. Wetter- oder Faxdienste. 

Anbieter webbasierter Anwendungen stehen jedoch vor dem Problem, dass komplexe Workflows abgebildet werden und diese auch persistent in der Anwendung verbleiben. Beispiele hierfür sind z.B. Werkzeuge zur projektspezifischen Zeiterfassung. Auch hier bietet sich die Möglichkeit der Anbindung mittels Schnittstellen, jedoch mit deutlich gesteigertem Aufwand, da zwischen beiden Parteien das Verständnis über die verarbeiteten Entitäten vermittelt werden muss. 

\begin{quote}
The idea of the Web services was born from the need to
provide a simple way of distributing and reusing a distant
application. Thus, the first generation of standards supporting
this technology was developed to meet primarily these aims.
The first standards do not cover certain aspects related to the
discovery, the composition and the selection of the services.\cite{ei-sawsdl}
\end{quote}

In dieser Seminararbeit möchte ich versuchen die Frage zu beantworten, welche Konzepte die dynamische Bindung von komplexen Webanwendungen ermöglichen, die für ein lebendiges Öko-System von entscheidender Bedeutung ist. Gibt es hierfür Ansätze und Lösungen?
\section*{Vorläufige Literaturliste}
\subsection*{zu Webservices allgemein}
\begin{itemize}
\item Web Site Evolution \cite{baresitalk2007}
\item Next Generation Semantic Web Applications \cite{Motta06nextgeneration}
\end{itemize}

\subsection*{zu Semantik}
\begin{itemize}
\item OIL: An Ontology Infrastructure for the Semantic Web \cite{oil}
\item On communication and coordination issues of Semantic Web Services \cite{WSMO}
\item Semantic SOA – Automatisierung und Interoperabilität in Service-Orientierten Architekturen \cite{semsoa}
\item SOA4All, Enabling the SOA Revolution on a World Wide Scale \cite{SOA4ALL}
\item The Role of Ontologies in Virtual Engineering \cite{ontologies}
\end{itemize}

\subsection*{zu Lösungsansätzen}
\begin{itemize}
\item WSMO-Lite: Lightweight Semantic Descriptions for Services on the Web \cite{WSMOLITE}
\item Sharing service semantics using SOAP-based and REST Web services \cite{shi1}
\item SAWSDL: Semantic Annotations for WSDL and XML Schema \cite{ky-sawsdl}
\item A metamodel of WSDL Web services using SAWSDL semantic annotations \emph{ei-sawsdl}
\item Semantic Annotations for WSDL and XML Schema. Recommendation W3C, 2007. \cite{w3c-sawsdl}
\item An Ontology-Based Resource-Oriented Information Supported Framework towards RESTful Service Generation and Invocation \cite{zg-ontorest}
\item SAPS: Semantic AtomPub-Based Services \cite{lr-saps}
\end{itemize}
\bibliography{bibliothek}
\bibliographystyle{plain} 
\end{document}