\documentclass[12pt]{article}
\usepackage[utf8]{inputenc}
\usepackage{german}
\usepackage[paper=a4paper]{geometry}
\begin{document}
\author{Markus Tacker}
\title{Konzepte und Standards zur domänenübergreifenden Integration von Webanwendungen}
\maketitle
\section*{Abstract}
Anbieter webbasierter Anwendungen stehen vor dem Problem der isolierten Existenz. In der Regel erfolgt die Anbindung zwischen Anwendungen über Schnittstellen, die den Dienst in Form einer sogenannten \emph{Black Box} veröffentlichen, d.h. dass der Dienst durch Übergabe eines Datums aufgerufen wird, dieser entsprechende des Aufrufes reagiert und ggfs. ein modifiziertes Datum zurück liefert. Beispiele hierfür sind z.B. Wetter- oder Faxdienste. Diese Arten von Services lassen sich über öffentliche Verzeichnisse finden.

Soll eine Integration der Dienste darüber hinaus erfolgen, muss zwischen den Parteien ein gemeinsames Verständnis über die verarbeiteten Entitäte geschaffen werden. Dies erfolgt in der Regel mit der individuellen Anbindung der Dienste. \emph{ontologies as a
neutral description}

Welche Konzepte aber ermöglichen die dynamische Bindung von komplexen Webanwendungen? Gibt es hierfür Ansätze und Lösungen?
\bibliography{bibliothek}
\bibliographystyle{plain} 
\end{document}