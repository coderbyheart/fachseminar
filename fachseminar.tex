%
% Fachseminararbeit
%
% @author: Markus Tacker <m@coderbyheart.de>
% 
\documentclass[10pt,a4paper]{article}
\usepackage[utf8]{inputenc}
\usepackage[german]{babel}
\usepackage[T1]{fontenc}
\usepackage{graphicx}

\usepackage[paper=a4paper,width=14cm,left=35mm,height=22cm]{geometry}
\usepackage{setspace}
\usepackage{acronym}
\usepackage{charter}
\linespread{1.15}
\setlength{\parskip}{0.25em}
\setlength{\parindent}{0em}

\bibliographystyle{plain} 

\bibdata{bibliothek}

% Abkürzungen werden hier definiert, da es kein "öffentliches" Abkürzungsverzeichnis gibt

\newacro{SOAP}[SOAP]{Simple Object Access Protocol}
\newacro{UDDI}[UDDI]{Universal Description, Discovery and Integration}
\newacro{ORISF}[ORISF]{Ontology-based Resourceoriented Information Supported Framework}
\newacro{SAWSDL}[SAWSDL]{Semantic Annotations for WSDL}
\newacro{SOA}[SOA]{Service Oriented Architecture}
\newacro{API}[API]{Application Programming Interface}

\begin{document}
\author{Markus Tacker}
\title{Konzepte und Standards zur domänenübergreifenden Integration von komplexen Webanwendungen}

\begin{center}

\begin{small}Fachseminararbeit · Wintersemester 2011/2012\\Fachbereich Design Informatik Medien · Hochschule RheinMain\\Bachelor-Studiengang Medieninformatik\end{small}

\bigskip

\begin{huge}Konzepte und Standards 
\medskip
zur domänenübergreifenden Integration
\medskip
von komplexen Webanwendungen\end{huge}

\bigskip

\begin{large}Markus Tacker\end{large}

\begin{small}\texttt{https://github.com/tacker/fachseminar/}\end{small}

\end{center}

\section*{Abstract}

Die Suche nach Konzepten für die dynamische Bindung von \acl{WS} ist die Motivation für diese Fachseminararbeit. Vorraussetzung dafür ist, dass \acl{WS} semantisch beschrieben werden, erst so ist eine automatische Dienstvermittlung zur Laufzeit möglich. Wie Dostal und Jeckle aber in \cite[S.55]{xmlspek1} beschreiben, wurden aktuell verbreitete Standards zur Anbindung von \acl{WS} wie z.B. die \acs{WSDL} aber im Hinblick auf die Anbindung von \emph{Services} mit einer konkreten \acs{API} entwickelt --- sie beschreiben den syntaktischen Rahmen einer Schnittstelle, das zu Grunde liegende Wissen über das Domänenkonzept, also die \emph{Semantik}, wird nicht festgehalten. 

Nach einer Einführung in das Thema in Abschnitt~\ref{l:einleitung}, in dem der aktuellen Stand der Entwicklung beschrieben wird und die daraus resultierende Hindernisse bei der dynamischen Bindung von \acl{WS} erläutert werden, werden in Abschnitt~\ref{l:sem-web-ser} die theoretischen Aspekte und wie man mit Hilfe von \emph{Ontologien} Domänenkonzepte semantisch beschreibbar machen kann vorgestellt. Mit der \acs{SAWSDL} liefert das \acs{W3C} den Entwurf eines Standards für diese Aufgabe, der am Ende dieses Abschnittes beschrieben wird.

Abschnitt~\ref{l:loesungen} stellt dann schließlich einen möglichen Lösungsansatz für die Implentierung einer \acs{SOA} vor, deren Dienste zur Laufzeit gebunden werden.

Im Abschnitt~\ref{l:verwendung} wird dann die Anwendbarkeit der aufgezeigten Konzepte auf den Bereich der \emph{komplexen Webanwendungen} überprüft, allem auszeichnet, dass sie im Gegensatz zu einfachen \acl{WS} zustandsbehaftet sind und ihre Integration mit den gängigen Mitteln nur individuell und statisch möglich ist.

\pagebreak

\tableofcontents

\section{Einleitung}
\label{l:einleitung}

Ein Teil der neueren Entwicklung des Internets zum \emph{Web 2.0} basiert auf der Idee, dass Informationen und Funktionen von Software mit Hilfe von \emph{Webservices} verwendet werden können.

Die Kommunikation mit Webservices ist zwar auf Protokollebene standardisiert, muss jedoch vom Konsumenten immer individuell entsprechend dem Domänenmodell des Anbieters implementiert werden, wodurch eine feste Bindung an den Anbieter entsteht \cite{ka-cots}.

Für sogenannte \emph{Blackbox-Webservices} ist das kein Problem --- diese \emph{zustandslosen} Dienste verarbeiten lediglich einfache Daten, d.h. dass der Dienst aufgerufen wird, dieser entsprechend des Aufrufs, z.B. anhand eines übergebenen Datums, reagiert und ein Ergebnis zurückliefert. Jede weitere Anfrage wird unabhängig von einer vorherigen behandelt.

Beispiele hierfür ist z.B. ein Webservice, der Wetterdaten für eine PLZ liefert. Hier gibt der Konsument die PLZ eines Ortes in Deutschland ein und erhält in der Antwort eine Temperatur. 

Für die Nutzung des Dienstes reicht die Kenntnis der Schnittstellen aus. Die genauen technischen Abläufe, wie der Webservice aus der PLZ eine Temperatur ermittelt bleiben für den Konsumenten verborgen und sind für diesen auch irrelevant \cite{hhxmlwssoa}.

Diese Eigenschaft steht in direktem Zusammenhang mit einem der wichtigsten Konzepte in der Softwareentwicklung der vergangenen Jahre: In einer \ac{SOA} werden Anwendungen nicht mehr monolithisch aufgebaut, sondern in kleinere, in sich geschlossene Komponenten unterteilt. Diese kommunizieren miteinander in einem Intra- oder ein Extranet über ihre öffentliche Schnittstellen, der sogenannten \ac{API}.

Diese Kapselung von Diensten hat auch zum Ziel, eine möglichst hohe Kohäsion innerhalb eines Systems zu ermöglichen --- Quellcode soll, wenn möglich, nur einmal geplant, entworfen und geschrieben werden, und im ganzen System verwendet werden, woraus im Endergebnis weniger Code, eine höhere Standardisierung und damit letztendlich niedrigere Kosten resultieren \cite{hn-web20}.

\emph{Webbasierter Anwendungen} zeichnet jedoch aus, dass sie komplexe Arbeitsabläufe abbilden --- die Anwendung wird dadurch \emph{zustandsbehaftet}. Als Beispiel für diese Art von webbasierten Diensten werde ich in dieser Seminararbeit die Online-Zeiterfassung \emph{mite}\footnote{http://mite.yo.lk/} betrachten, mit dem man Arbeitszeit Erfassen und Auswertung kann.

Die Funktionalität von \emph{mite} kann für sich alleinstehend verwendet werden. Hierzu werden über die Website die nötigen Businessdaten von \emph{mite} (Benutzer, Leistungen\footnote{beschreibt eine Tätigkeit, z.B. Programmierung, mit einem Stundensatz}, Projekte, Kunden und Zeiten) angelegt. Diese Daten werden bei \emph{mite} gespeichert. 

Das Problem bei Schnittstellen zu zustandsbehaften Webservices resultiert daraus, dass der Verwender des Services eine Vermittlungsschicht zwischen seiner Domänen-Logik und der des Webservices implementieren muss, die zwischen beiden Parteien das Verständnis über die verarbeiteten Entitäten vermittelt und er sich so fest an den jeweiligen Dienst bindet.

Angenommen, ein IT-Unternehmen setzt in seinem Intranet eine Software zur Projektverwaltungssoftware ein, in dem alle Mitarbeiter auf alle Projekte, an denen sie arbeiten zugriff haben. In diesem Intranet werden auch die Aufgaben zu den einzelnen Projekten verwaltet, sowie die zugeordneten Kostenstellen. Die Software bietet auch eine Funktion zur Zeiterfassung an, diese ist aber sehr unkomfortabel und fehlerhaft und wird von den Mitarbeitern deswegen nicht verwendet. 

Das Unternehmen entscheidet sich nun, \emph{mite} zur Zeiterfassung einzusetzen. Mit dessen öffentlicher API\footnote{http://mite.yo.lk/api/index.html} ist es möglich, alle Businessdaten zu bearbeiten. Um die Verwendung für die eigenen Mitarbeiter so komfortabel wie möglich zu machen, und die erfassten Zeiten automatisch den eigenen Projekten zuordnen zu können, implementiert das Unternehmen in der Intranet-Software ein Mapping zwischen seinen eigenen Businessdaten und denen von \emph{mite}. Mitarbeiter entsprechen dabei Benutzern, die rechnerischen Stundensätzen von Kostenstellen entsprechen Leistungen. Projekte, Kunden und Zeiten existieren zwar in beiden Domänen, aber mit gänzlich unterschiedlichen Attributen --- auch hierfür ist ein Mapping notwendig. Mit Hilfe das Mappings können beide System parallel verwendet werden und es ist sichergestellt, dass die Datenbestände beider Seiten synchronisiert sind.

\textbf{Was ist aber in dem Fall, dass der gewählte Service nicht mehr eingesetzt werden soll oder kann?}

Nach Elyacoubi, Belouadha und Roudies in \cite{ei-sawsdl} sind etablierte Standards für Webservices der ersten Generation wie \ac{SOAP} und \ac{UDDI} primär unter dem Aspekt entwickelt worden, einen einfachen Weg zur Verteilung und Wiederverwertung von Webservices zu etablieren --- ihnen fehlt also eine Standardisierung für das Auffinden, Zusammenstellen und Auswählen von Diensten um eine \emph{lose Kopplung} zu ermöglichen. 

\framebox[\linewidth]{\parbox{0.98\linewidth}{
However, these standards are not sufficient to allow an automation of the various tasks of the Web service’s life cycle, namely the discovery, the invocation, the publication and the composition. Recently, the W3C consortium produced the SAWSDL language. \cite[S.653]{ei-sawsdl}
}}

\label{l:intro-loosecoupling}Für ein lebendiges Web-Öko-System ist die lose Kopplung jedoch von entscheidender Bedeutung --- im Idealfall lassen sich Dienste so anbinden, dass sie jederzeit und mit geringem Aufwand ausgetauscht werden können. 

Das Mapping zwischen den Diensten müsste also so allgemein definiert sein, dass lediglich die \emph{Definition} angepasst werden müsste, aber nicht die \emph{Implementierung} --- ein Wechseln des Services hätte lediglich das Ändern einer Schnittstellenbeschreibung zur Folge, ohne dass man konkreten Quellcode anpassen muss.

Führt man diesen Gedanken noch einen Schritt weiter, wäre es optimal, wenn selbst der Schritt des Erstellen des Mappings automatisiert erfolgen kann, da sich die zum Einsatz kommenden Systeme über die Begrifflichkeiten, die der jeweils andere verwendet im Klaren sind \cite{mkdigioe}.

Im Hinblick auf ökonomische Aspekte kann es sogar von Vorteil sein, die parallele Verwendung mehrere Dienste der gleichen Art zu ermöglichen. Patel beschreibt z.B. in \cite{pl-depintra}, dass Unternehmensintranets oft zu unspezifisch für die individuellen Bedürfnisse eines einzelnen Mitarbeiters sind. Im unserem Intranet-Beispiel könnte das Unternehmen seinen Mitarbeitern die Zeiterfassung mit \emph{mite} ermöglichen, aber auch alternativ mit z.B. \emph{TimeNote}\footnote{http://www.timenote.de/}. Auf den ersten Blick erscheint diese Heterogenität kontraproduktiv, das Unternehmen eröffnet seinen Mitarbeitern aber so die Möglichkeit, das Werkzeug für die gegebene Aufgabe "`Zeiterfassung"' zu verwenden, dass ihren  Vorlieben am ehesten entspricht, und kann dadurch die Akzeptanz dafür steigern.

Neben der Möglichkeit der Wahl, erhält man so auch automatisch Redundanz.

Masak hat in \cite{mkulss} diesen Gedanken auf eine größere Ebene übertragen und kommt zu dem Schluss, dass es in einem digitalen Ökosystem, wie das Internet eines ist, notwendig ist, Redundanz auf allen Ebenen einzuführen, und durch eine abstrahiertes Mapping eine ad-hoc Komposition von Services zu ermöglichen.



\emph{Semantische Webservices} sind ein Konzept, mit dem es möglich wird, die Anbindung von Webservices abstrakt zu beschreiben und so eine lose Kopplung zu erreichen.

In diesem Abschnitt erläutere ich deren Grundlagen.

Wie schon auf Seite \pageref{l:intro-loosecoupling} beschrieben, ist Voraussetzung für eine Service-Infrastruktur mit loser Kopplung, dass die Bedeutung der Aufgabe, die mit dem Webservice abgebildet wird automatisch ermittelt werden kann.\cite{xmlspek1}

Beschreibt man einen Webservice mittels \ac{WSDL}, legt man damit lediglich den Syntax für die vom Webservice verarbeiteten Anfragen fest, die Bedeutung der Funktionalität und der übertragenen Daten erschließt sich daraus nicht, sondern entsteht lediglich in der Interpretation der Benutzer des Dienstes.

Es fehlt also eine Komponente, die dem reinen Akt der Datenübertragung ein inhaltlichen Beschreibung hinzufügt.

In unserem Beispiel mit dem Intranet und der Zeiterfassung, z.B. die Information, dass Mitarbeiter des Unternehmens eine Entsprechung in \ac{mite} haben, nämlich dort als Benutzer verstanden werden.

\subsection{Semantik}

\emph{Semantik}

\paragraph{Weiterführende Publikationen} In einer Artikelserie aus dem Jahr 2004 (\cite{xmlspek1}, \cite{xmlspek2}, \cite{xmlspek3} und \cite{xmlspek4}) beschreiben Wolfgang Dostal, Mario Jeckle und Werner Kriechbaum ausführlich das Konzept der semantischen Webservices.

Ziel einer Lösung muss es also sein, dass man Web Services, die zur Entwicklungszeit noch unbekannt sind, dynamisch binden kann. In diesem Abschnitt stelle ich Lösungsansätze vor, in denen die dynamische Bindung von semantischen \ws möglich ist.

\subsection{Lösungsansätze}

Wie Dostal und Jeckle in \cite[S.61]{xmlspek4} ausführen, werden in den üblichen Beschreibungen zum Ablauf in einem Web-Service-Szenario \ac{WSDL}-Dokumente hauptsächlich als Eingabe für einen Generator beschrieben, mit dessen Hilfe die programmierspezifischen Implementierungen erzeugt werden. Dies erfolgt allerdings in der Regel zur Entwicklungszeit der Anwendung und nicht zu deren Laufzeit. Zwar ist es bei Sprach- und Ausführungsumgebungen wie Java heute technisch durchaus möglich, auch zur Ausführungszeit Klassen der Anwendung hinzuzufügen und auf diesem Weg das oben beschriebene Implementierungszenario von der Entwicklungs- in die Laufzeit zu verlagern. Allerdings würde ein solcher Ansatz eine Reihe von Nachteilen bzw. Riksiken bergen. Das gewichtigste Argument gegen den Genierungsansatz ist zweifelsfrei im Bereich Sicherheit angesiedelt. Das Einbinden von nicht getestetem Code bietet geschickten Angreifer ein \emph{el Dorada} von Möglichkeiten, potentiell gefährliche Programmsequenzen in die Anwendung ein zu schmuggeln. Statt des generativen Ansatzes eignet sich daher ein Framework, das mit Hilfe der Informationen in einem \ac{WSDL}-Dokument eine \ac{SOAP}-Kommunikation durchführen kann, ohne dazu Codegenerierungen durchführen zu müssen, besser. Für Java-Anwendungen existiert dazu unter anderem das \ac{WSIF} der Apache Group. Entsprechend den Elementen einer WSDL-Beschreibung sind innerhalb des \ac{WSIF} Klassen definiert, die mittels der \ac{WSDL}-Eingabe parametrisiert werden. Damit ist es möglich jeden denkbaren Web Service "`spontan"' zu nutzen und so tatsächlich dynamisches Verhalten der Anwendungen in Web-Service-Szenarien zu erreichen. Abbildung~\ref{f:swcs} auf Seite~\pageref{f:swcs} zeigt schematisch die Ablaufe innerhalb einer solchen Lösung.

\begin{figure}[ht]
\centering
\parbox{0.85\textwidth}{
    \includegraphics[width=0.85\textwidth]{media/semantic-ws-client-system.png}
    \caption{\emph{Abläufe in einem System für semantische \wss} \cite[S.64]{ky-sawsdl}. Alle Abläufe, beginnend mit dem Auffinden von \wss die zu einer Aufgabe passen bis zum Aufrufen des ausgewählten \ws , können mit semantischen Technologien automatisiert werden.}
    \label{f:swcs}
}
\end{figure}

\subsection{Implementierungen}

Im Rahmen des \ac{ADDO}-Projektes\footnote{http://www.vs.uni-kassel.de/ADDO/index.html} and der Universität Kassel, dass sich zum Ziel gesetzt hat, einen automatischen Algorithmus zur qualitätsberücksichtigenden Service-Aufindung und ein Framework zur automatischen Serviceintegration und -verwaltung zu entwickeln haben Bleul, Zapf und Geihs in \cite[S.410ff]{flexbrok} eine Archtitektur entworfen, die in der Lage ist, die angesprochenen Anforderungen zu erfüllen. Die vorgestellte Architektur enthält einen \emph{Service Broker}, an dem sich semantische Services registrieren und der in der Lage ist automatisch Services aufzufinden. Ein \emph{Service Container} überwacht die Services und deren Integration in das System. Die Architektur kann Services auch zur Laufzeit austauschen und sich so selbst zu "`heilen"'. Anbieter von Services müssen in diesem System die semantische und syntaktische Beschreibung selber Erstellen und beim \emph{Service Broker} registrieren --- sie müssen sich auch um die Deregistrierung kümmern, sollte der Dienst nicht mehr zur Verfügung stehen \cite[S.416]{flexbrok}.

Auch eine Gruppe von Wissenschaftlern aus Italien hat mit dem \emph{C-Cube Framework}\cite{ccube} einen ähnlichen Entwurf vorgelegt. Das System besteht, ähnlich wie das \ac{ADDO}-Projekt aus Komponenten zur Verwaltung der aktiven Services (namentlich \emph{Service, Service Description, Service Monitoring}) zum Binden und Ausführen von Services (hier \emph{Service Execution}). Das System ist in der Lage automatisch auf Basis semantischer Beschreibungen nach zu einer Anfrage passenden Diensten zu suchen, diese zu Binden und auszuführen \cite[S.4]{ccube}.

\begin{figure}[ht]
\centering
\parbox{0.85\textwidth}{
    \includegraphics[width=0.85\textwidth]{media/addo-player-example.png}
    \caption{\emph{Referenz-Implementation des \ac{ADDO}-Projektes} \cite[S.418]{flexbrok}.}
    \label{f:addo-player}
}
\end{figure}

Das \ac{ADDO}-Projekt hat zur Verdeutlichung der Zusammenarbeit der vorgestellten Komponenten eine einfache Beispiel-Anwendung implementiert (siehe Abbildung~\ref{f:addo-player} auf Seite~\pageref{f:addo-player}). Die Anwendung ist ein einfaches Multimedia-Player-Frontend, das in Java implementiert ist. Der Player erfordert einen Service, der die Operationen \emph{Play}, \emph{Stop} und \emph{NextTitle} zur Verfügung stellt, sowie optional nach zusätzlich \emph{Pause} oder \emph{PreviousTitle}. Es stehen zwei Systeme zur Verfügung, die diese Operationen anbieten: ein \emph{PowerPoint}-Laptop, der an einen Beamer angeschlossen ist, sowie ein \emph{WinAmp}-Medien-Player, der an einer Stereo-Anlage angeschlossen ist. Beide System sind über einen eigenen \ws ansprechbar und implementieren mindestens die Operationen \emph{Play}, \emph{Stop} und \emph{NextTitle}. Ihre semantische Beschreibung liegt als \ac{OWL-S} im \emph{Service Broker} vor. Der \emph{Service Container} ist als Java-Bibliothek implementiert und steht in der Anwendung als Instanz zur Verfügung. Ziel ist es nun, einen generischen Medien-Player an die in der aktuellen Umgebung verfügbaren Operationen zu binden. Sobald der \emph{Service Container} von der Anwendung aufgerufen wird, löst die bis jetzt leer Bindung den \emph{Broker}-Mechanismus aus und entweder der \emph{PowerPoint}- oder der \emph{WinAmp}-Webservice wird durch den \emph{Service Broker} gebunden. Der gebundene wird an den \emph{Service Container} übergeben. Mit Hilfe eines \emph{ant}-Scripts auf Basis von \emph{WSDL2JAVA} (Teil des Axis-Projekts\footnote{http://ws.apache.org/axis/}) wird automatisch ein passender Service-Stub generiert, der die verfügbaren Operationen entsprechend der Webservice-Beschreibung, die in \ac{WSDL} vorliegt, konfiguriert und anschließend in den \emph{Service Container} integriert. Im Fall einer Service-Änderung (Wegfall eines gebundenen Services) wird eine Exception geworfen und die Anwendung kann entweder die Verwendung abbrechen oder einen neuen Service verwenden.

Die hier genannten Beispiel zeigen, dass es bereits funktionsfähige Implementierungen einer Architektur gibt, die in der Lage ist, Services zur Laufzeit zu binden.

\section{Anwendung bei komplexen webbasierten Anwendungen}
\label{l:verwendung}

% Literatur finden, die die Anwendung der o.g. Techniken beschreibt
% Dann können auch die Beispiele entfallen

\begin{itemize}
\item Webintents
\item OpenSearch
\end{itemize}

\section{Fazit}
\label{l:fazit}
\section{Ausblick}
\label{l:ausblick}
\pagebreak

\bibliography{bibliothek}
\end{document}
