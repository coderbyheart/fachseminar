\section*{Abstract}

Die Integration von Webservices erfolgt in der Regel durch das Entwickeln von Adaptern in der Umgebung, in der die Integration erfolgen soll. In dieser Fachseminararbeit werden Standards und Konzepte vorgestellt, die diesen Vorgang weitestgehend automatisieren können. Damit eine automatische Dienstvermittlung zur Laufzeit möglich wird, müssen Webservices semantisch beschrieben werden. Wie Dostal und Jeckle aber in \cite[S.55]{xmlspek1} beschreiben, wurden Standards zur Anbindung von Webservices wie z.B. \acs{WSDL} im Hinblick auf die Anbindung von \emph{Services} mit einer konkreten API entwickelt --- sie beschreiben den syntaktischen Rahmen einer Schnittstelle, das zu Grunde liegende Wissen über das Domänenkonzept, also die \emph{Semantik}, wird jedoch nicht beschrieben. Nach einer Einführung in das Thema in Abschnitt~\ref{l:einleitung}, in dem der aktuelle Stand der Entwicklung beschrieben wird und die daraus resultierende Hindernisse bei der dynamischen Bindung von Webservice erläutert werden, führt Abschnitt~\ref{l:sem-web-ser} in die theoretischen Aspekte ein und zeigt wie man mit Hilfe von \emph{Ontologien} Domänenkonzepte semantisch beschreibbar machen kann. Mit der \acs{SAWSDL} existiert ein Standard für diese Aufgabe. Abschnitt~\ref{l:loesungen} stellt Lösungsansätze für die Implementierung einer \acs{SOA} vor, deren Dienste zur Laufzeit gebunden werden. Im Fazit in Abschnitt~\ref{l:fazit} werden die aufgezeigten Konzepte auf ihre Anwendbarkeit auf \emph{komplexen Webanwendungen} analysiert.