\section*{Abstract}

Die Integration von Webservices erfolgt in der Regel durch das Entwickeln von Adaptern in der Umgebung, in der die Integration erfolgen soll. In dieser Fachseminararbeit stelle ich Standards und Konzepte vor, die diesen Vorgang weitestgehend automatisieren können. Voraussetzung dafür ist, dass Webservices semantisch beschrieben werden, erst dann ist eine automatische Dienstvermittlung zur Laufzeit möglich. Wie Dostal und Jeckle aber in \cite[S.55]{xmlspek1} beschreiben, wurden jedoch Standards zur Anbindung von Webservices wie z.B. \acs{WSDL} im Hinblick auf die Anbindung von \emph{Services} mit einer konkreten \acs{API} entwickelt --- sie beschreiben den syntaktischen Rahmen einer Schnittstelle, das zu Grunde liegende Wissen über das Domänenkonzept, also die \emph{Semantik}, wird nicht beschrieben. Nach einer Einführung in das Thema in Abschnitt~\ref{l:einleitung}, in dem der aktuelle Stand der Entwicklung beschrieben wird und die daraus resultierende Hindernisse bei der dynamischen Bindung von Webservice erläutert werden, führt Abschnitt~\ref{l:sem-web-ser} in die theoretischen Aspekte ein und zeigt wie man mit Hilfe von \emph{Ontologien} Domänenkonzepte semantisch beschreibbar machen kann. Mit der \acs{SAWSDL} existiert ein Standards für diese Aufgabe, der in Abschnittes~\ref{l:sawsdl} beschrieben wird. Abschnitt~\ref{l:loesungen} stellt mögliche Lösungsansätze für die Implentierung einer \acs{SOA} vor, deren Dienste zur Laufzeit gebunden werden. Im Fazit in Abschnitt~\ref{l:fazit} werden die aufgezeigten Konzepte auf ihre Anwendbarkeit auf \emph{komplexen Webanwendungen} analysiert.