\section*{Abstract}

Die Integration von Dienstangeboten über das Internet als sogenannte 
\emph{Webservices} ist heutzutage kein Problem mehr. Existierende Standards
wie z.B. \ac{SOAP} und \ac{UDDI} liefern hierfür bewährte Werkzeuge.

Diese Standards der ersten Generation wurden jedoch im Hinblick auf 
die Anbindung zustandsloser Webservices entwickelt~\cite[S. 653]{ei-sawsdl} 
und bilden die Verbindung zwischen zwei Diensten immer individuell ab.

Hieraus ergibt sich jedoch ein Problem bei der Anbindung komplexer 
Webanwendungen: werden diese mit Hilfe der genannten Techniken 
angebunden, wird das zur Vermittlung zwischen den jeweiligen Domänenkonzepten
nötige Wissen in der Implementierung der Integration \emph{hart kodiert} --- 
andere oder zusätzliche Dienste gleicher Art können deswegen nicht ohne erneuten
Aufwand angebunden werden.

Die Suche nach Konzepten für die dynamische Bindung von komplexen Webanwendungen
ist die Motivation für diese Fachseminararbeit, in der ich in Abschnitt~\ref{l:sem-web-ser} die Möglichkeit vorstelle, Webservices semantisch zu beschreiben. Neben einer Einführung in das Thema, in der ich auf die theoretischen Aspekte eingehe, analysiere ich in Abschnitt~\ref{l:loesungen} Umsetzungen dieser Theorien in den Standards \ac{SAWSDL} und \ac{ORISF}.

Im \ref{l:verwendung}. Abschnitt beurteile ich dann deren Anwendung bei der Anbindung 
komplexer Webanwendungen, in dem ich beispielhafte Implementierungen aus der Praxis aufzeige.

