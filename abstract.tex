\section*{Abstract}

Die Suche nach Konzepten für die dynamische Bindung von \acl{WS} ist die Motivation für diese Fachseminararbeit. Vorraussetzung dafür ist, dass \acl{WS} semantisch beschrieben werden, erst so ist eine automatische Dienstvermittlung zur Laufzeit möglich. Wie Dostal und Jeckle aber in \cite[S.55]{xmlspek1} beschreiben, wurden aktuell verbreitete Standards zur Anbindung von \acl{WS} wie z.B. die \acs{WSDL} aber im Hinblick auf die Anbindung von \emph{Services} mit einer konkreten \acs{API} entwickelt --- sie beschreiben den syntaktischen Rahmen einer Schnittstelle, das zu Grunde liegende Wissen über das Domänenkonzept, also die \emph{Semantik}, wird nicht festgehalten. 

Nach einer Einführung in das Thema in Abschnitt~\ref{l:einleitung}, in dem der aktuellen Stand der Entwicklung beschrieben wird und die daraus resultierende Hindernisse bei der dynamischen Bindung von \acl{WS} erläutert werden, werden in Abschnitt~\ref{l:sem-web-ser} die theoretischen Aspekte und wie man mit Hilfe von \emph{Ontologien} Domänenkonzepte semantisch beschreibbar machen kann vorgestellt. Mit der \acs{SAWSDL} liefert das \acs{W3C} den Entwurf eines Standards für diese Aufgabe, der am Ende dieses Abschnittes beschrieben wird.

Abschnitt~\ref{l:loesungen} stellt dann schließlich einen möglichen Lösungsansatz für die Implentierung einer \acs{SOA} vor, deren Dienste zur Laufzeit gebunden werden.

Im Abschnitt~\ref{l:verwendung} wird dann die Anwendbarkeit der aufgezeigten Konzepte auf den Bereich der \emph{komplexen Webanwendungen} überprüft, allem auszeichnet, dass sie im Gegensatz zu einfachen \acl{WS} zustandsbehaftet sind und ihre Integration mit den gängigen Mitteln nur individuell und statisch möglich ist.