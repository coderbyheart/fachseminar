\section{Einleitung}
\label{l:einleitung}

Ein Teil der neueren Entwicklung des Internets zum \emph{Web 2.0} basiert auf der Idee, dass Informationen und Funktionen von Software mit Hilfe von \emph{Webservices} verwendet werden können.

Die Kommunikation mit Webservices ist zwar auf Protokollebene standardisiert, muss jedoch vom Konsumenten immer individuell entsprechend dem Domänenmodell des Anbieters implementiert werden, wodurch eine feste Bindung an den Anbieter entsteht \cite{ka-cots}.

Für sogenannte \emph{Blackbox-Webservices} ist das kein Problem --- diese \emph{zustandslosen} Dienste verarbeiten lediglich einfache Daten, d.h. dass der Dienst aufgerufen wird, dieser entsprechend des Aufrufs, z.B. anhand eines übergebenen Datums, reagiert und ein Ergebnis zurückliefert. Jede weitere Anfrage wird unabhängig von einer vorherigen behandelt.

Beispiele hierfür ist z.B. ein Webservice, der Wetterdaten für eine PLZ liefert. Hier gibt der Konsument die PLZ eines Ortes in Deutschland ein und erhält in der Antwort eine Temperatur. 

Für die Nutzung des Dienstes reicht die Kenntnis der Schnittstellen aus. Die genauen technischen Abläufe, wie der Webservice aus der PLZ eine Temperatur ermittelt bleiben für den Konsumenten verborgen und sind für diesen auch irrelevant \cite{hhxmlwssoa}.

Diese Eigenschaft steht in direktem Zusammenhang mit einem der wichtigsten Konzepte in der Softwareentwicklung der vergangenen Jahre: In einer \ac{SOA} werden Anwendungen nicht mehr monolithisch aufgebaut, sondern in kleinere, in sich geschlossene Komponenten unterteilt. Diese kommunizieren miteinander in einem Intra- oder ein Extranet über ihre öffentliche Schnittstellen, der sogenannten \ac{API}.

Diese Kapselung von Diensten hat auch zum Ziel, eine möglichst hohe Kohäsion innerhalb eines Systems zu ermöglichen --- Quellcode soll, wenn möglich, nur einmal geplant, entworfen und geschrieben werden, und im ganzen System verwendet werden, woraus im Endergebnis weniger Code, eine höhere Standardisierung und damit letztendlich niedrigere Kosten resultieren \cite{hn-web20}.

\emph{Webbasierter Anwendungen} zeichnet jedoch aus, dass sie komplexe Arbeitsabläufe abbilden --- die Anwendung wird dadurch \emph{zustandsbehaftet}. Als Beispiel für diese Art von webbasierten Diensten werde ich in dieser Seminararbeit die Online-Zeiterfassung \emph{mite}\footnote{http://mite.yo.lk/} betrachten, mit dem man Arbeitszeit Erfassen und Auswertung kann.

Die Funktionalität von \emph{mite} kann für sich alleinstehend verwendet werden. Hierzu werden über die Website die nötigen Businessdaten von \emph{mite} (Benutzer, Leistungen\footnote{beschreibt eine Tätigkeit, z.B. Programmierung, mit einem Stundensatz}, Projekte, Kunden und Zeiten) angelegt. Diese Daten werden bei \emph{mite} gespeichert. 

Das Problem bei Schnittstellen zu zustandsbehaften Webservices resultiert daraus, dass der Verwender des Services eine Vermittlungsschicht zwischen seiner Domänen-Logik und der des Webservices implementieren muss, die zwischen beiden Parteien das Verständnis über die verarbeiteten Entitäten vermittelt und er sich so fest an den jeweiligen Dienst bindet.

Angenommen, ein IT-Unternehmen setzt in seinem Intranet eine Software zur Projektverwaltungssoftware ein, in dem alle Mitarbeiter auf alle Projekte, an denen sie arbeiten zugriff haben. In diesem Intranet werden auch die Aufgaben zu den einzelnen Projekten verwaltet, sowie die zugeordneten Kostenstellen. Die Software bietet auch eine Funktion zur Zeiterfassung an, diese ist aber sehr unkomfortabel und fehlerhaft und wird von den Mitarbeitern deswegen nicht verwendet. 

Das Unternehmen entscheidet sich nun, \emph{mite} zur Zeiterfassung einzusetzen. Mit dessen öffentlicher API\footnote{http://mite.yo.lk/api/index.html} ist es möglich, alle Businessdaten zu bearbeiten. Um die Verwendung für die eigenen Mitarbeiter so komfortabel wie möglich zu machen, und die erfassten Zeiten automatisch den eigenen Projekten zuordnen zu können, implementiert das Unternehmen in der Intranet-Software ein Mapping zwischen seinen eigenen Businessdaten und denen von \emph{mite}. Mitarbeiter entsprechen dabei Benutzern, die rechnerischen Stundensätzen von Kostenstellen entsprechen Leistungen. Projekte, Kunden und Zeiten existieren zwar in beiden Domänen, aber mit gänzlich unterschiedlichen Attributen --- auch hierfür ist ein Mapping notwendig. Mit Hilfe das Mappings können beide System parallel verwendet werden und es ist sichergestellt, dass die Datenbestände beider Seiten synchronisiert sind.

\textbf{Was ist aber in dem Fall, dass der gewählte Service nicht mehr eingesetzt werden soll oder kann?}

Nach Elyacoubi, Belouadha und Roudies in \cite{ei-sawsdl} sind etablierte Standards für Webservices der ersten Generation wie \ac{SOAP} und \ac{UDDI} primär unter dem Aspekt entwickelt worden, einen einfachen Weg zur Verteilung und Wiederverwertung von Webservices zu etablieren --- ihnen fehlt also eine Standardisierung für das Auffinden, Zusammenstellen und Auswählen von Diensten um eine \emph{lose Kopplung} zu ermöglichen. 

\framebox[\linewidth]{\parbox{0.98\linewidth}{
However, these standards are not sufficient to allow an automation of the various tasks of the Web service’s life cycle, namely the discovery, the invocation, the publication and the composition. Recently, the W3C consortium produced the SAWSDL language. \cite[S.653]{ei-sawsdl}
}}

\label{l:intro-loosecoupling}Für ein lebendiges Web-Öko-System ist die lose Kopplung jedoch von entscheidender Bedeutung --- im Idealfall lassen sich Dienste so anbinden, dass sie jederzeit und mit geringem Aufwand ausgetauscht werden können. 

Das Mapping zwischen den Diensten müsste also so allgemein definiert sein, dass lediglich die \emph{Definition} angepasst werden müsste, aber nicht die \emph{Implementierung} --- ein Wechseln des Services hätte lediglich das Ändern einer Schnittstellenbeschreibung zur Folge, ohne dass man konkreten Quellcode anpassen muss.

Führt man diesen Gedanken noch einen Schritt weiter, wäre es optimal, wenn selbst der Schritt des Erstellen des Mappings automatisiert erfolgen kann, da sich die zum Einsatz kommenden Systeme über die Begrifflichkeiten, die der jeweils andere verwendet im Klaren sind \cite{mkdigioe}.

Im Hinblick auf ökonomische Aspekte kann es sogar von Vorteil sein, die parallele Verwendung mehrere Dienste der gleichen Art zu ermöglichen. Patel beschreibt z.B. in \cite{pl-depintra}, dass Unternehmensintranets oft zu unspezifisch für die individuellen Bedürfnisse eines einzelnen Mitarbeiters sind. Im unserem Intranet-Beispiel könnte das Unternehmen seinen Mitarbeitern die Zeiterfassung mit \emph{mite} ermöglichen, aber auch alternativ mit z.B. \emph{TimeNote}\footnote{http://www.timenote.de/}. Auf den ersten Blick erscheint diese Heterogenität kontraproduktiv, das Unternehmen eröffnet seinen Mitarbeitern aber so die Möglichkeit, das Werkzeug für die gegebene Aufgabe "`Zeiterfassung"' zu verwenden, dass ihren  Vorlieben am ehesten entspricht, und kann dadurch die Akzeptanz dafür steigern.

Neben der Möglichkeit der Wahl, erhält man so auch automatisch Redundanz.

Masak hat in \cite{mkulss} diesen Gedanken auf eine größere Ebene übertragen und kommt zu dem Schluss, dass es in einem digitalen Ökosystem, wie das Internet eines ist, notwendig ist, Redundanz auf allen Ebenen einzuführen, und durch eine abstrahiertes Mapping eine ad-hoc Komposition von Services zu ermöglichen.
