Ziel einer Lösung muss es also sein, dass man Web Services, die zur Entwicklungszeit noch unbekannt sind, dynamisch binden kann. Wie Dostal und Jeckle in \cite[S.61]{xmlspek4} ausführen, werden in den üblichen Beschreibungen zum Ablauf in einem Web-Service-Szenario \ac{WSDL}-Dokumente hauptsächlich als Eingabe für einen Generator beschrieben, mit dessen Hilfe die programmierspezifischen Implementierungen erzeugt werden. Dies erfolgt allerdings in der Regel zur Entwicklungszeit der Anwendung und nicht zu deren Laufzeit.

Zwar ist es bei Sprach- und Ausführungsumgebungen wie Java heute technisch durchaus möglich, auch zur Ausführungszeit Klassen der Anwendung hinzuzufügen und auf diesem Weg das oben beschriebene Implementierungszenario von der Entwicklungs- in die Laufzeit zu verlagern. Allerdings würde ein solcher Ansatz eine Reihe von Nachteilen bzw. Riksiken bergen. Das gewichtigste Argument gegen den Genierungsansatz ist zweifelsfrei im Bereich Sicherheit angesiedelt. Das Einbinden von nicht getestetem Code bietet geschickten Angreifenr ein \emph{el Dorada} von Möglichkeiten, potentiell gefährliche Programmsequenzen in die Anwendung ein zu schmuggeln.

Statt des generativen Ansatzes eignet sich daher ein Framework, das mit Hilfe der Informationen in einem \ac{WSDL}-Dokument eine \ac{SOAP}-Kommunikation durchführen kann, ohne dazu Codegenerierungen durchführen zu müssen, besser. Für Java-Anwendungen existiert dazu unter anderem das \ac{WSIF} der Apache Group. Entsprechend den Elementen einer WSDL-Beschreibung sind innerhalb des \ac{WSIF} Klassen definiert, die mittels der \ac{WSDL}-Eingabe parametrisiert werden. Damit ist es möglich jeden denkbaren Web Service "`spontan"' zu nutzen und so tatsächlich dynamisches Verhalten der Anwendungen in Web-Service-Szenarien zu erreichen.


% TODO: Abbildung 4 aus \cite{ky-sawsdl} rein


\subsection{Dynamische Verwendung von semantischen Webservices}

The proposed architecture contains a Service Broker to
register semantic service descriptions and perform automatic
service discovery, and a Service Container for monitoring
services and service integration. The process of automatic
service replacement is achieved by an interaction of service
brokering request, service discovery, and service integration
between Service Broker and Service Container. \cite[S.410]{flexbrok}

The specification
and publishing of service descriptions, either semantic
or syntactic, is done manually. ... The specification
of the semantic service description of a service offer and its
publication is manually done by the service vendor. ... Furthermore, service descriptions must be deleted at the
Service Broker if they become unavailable. \cite[S.416]{flexbrok}

Beispiel: Media-Player \cite[S.418]{flexbrok}

Weiterer Artikel: \cite{ccube}

