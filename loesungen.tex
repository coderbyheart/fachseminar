\ac{RDF} als Möglichkeit, Semantik in XML zu beschreiben.

In this paper we will explore this issue in some detail and we will propose a set of
features that, in our view, will increasingly characterize the Semantic Web
applications. Our analysis aims to be both descriptive and prescriptive. Descriptively,
the objective here is to characterize the space of current Semantic Web applications,
provide dimensions to compare and contrast them, and identify key trends.
Prescriptively, our goal is to specify a number of criteria, which Semantic Web
applications ought to satisfy, if we want to move away from conventional semantic
systems and develop a new generation of Semantic Web applications, which can
succeed in applying semantic technology to the challenging context provided by the
World-Wide-Web. \cite{ngswa}

The current Web service standards around Universal
Description, Discovery and Integration (UDDI) and related technologies do not address this
problem as they offer only service discovery based on attribute/ value queries, that is limited
to atomic service discovery. Research has shown that Business Process Execution Language
for Web Services (BPEL4WS or BPEL for short), which is currently used to express business
processes in Web service environments, does not have a solid formal model and thus lacks
formal semantics for querying business process descriptions. This means that a formal model is
required to express business processes and to enable their querying. Based on this formal model,
appropriate indexing techniques are needed for effcient querying in large service repositories. \cite{mothesis}

\subsection{SAWSDL}

As we noted in section 1, the technology of the Web
services was accompanied by many standards. However,
these standards do not remedy to the problem of adaptability
to Web services’ changes, and do not cover all aspects related
to different tasks of the Web service’s life cycle, namely, the
discovery, the invocation, the publication and the
composition. Indeed, many enterprises applications, in
particular, can constantly have need to discover and use
existing Web services, or to compose them to meet new
requirements or a complex request (eg. a travel agency can
for example use both fly and hotel booking services to
respond a customer’s query). The number of Web services
which increase on Internet makes their discovery and/or
composition a complex task. Automating these tasks is a
solution which would reduce the cost of the Web services’
implementation. Nevertheless, with this intention, we think
that it is necessary to enrich the description of the Web
services by explicit and comprehensible semantics, which can
be used by the machine. \cite{ei-sawsdl}

The more valuable
gain of SAWSDL lies in opportunities to annotate existing
WSDL descriptions in a bottom-up fashion while at the
same time only use descriptions of services which are relevant
to specific domain requirements. \cite{WSMOLITE}

\subsection{ORISF}

This paper presents the Ontology-based Resourceoriented
Information Supported Framework (ORISF). The
framework’s aim is to support RESTful Web Service with
resource model through ontology evolution and RESTful
service description, and realize the semantic-level
integration of resources. \cite{zg-ontorest}
