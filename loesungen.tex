Ziel einer Lösung muss es also sein, dass man Web Services, die zur Entwicklungszeit noch unbekannt sind, dynamisch binden kann. Wie Dostal und Jeckle in \cite[S.61]{xmlspek4} ausführen, werden in den üblichen Beschreibungen zum Ablauf in einem Web-Service-Szenario \ac{WSDL}-Dokumente hauptsächlich als Eingabe für einen Generator beschrieben, mit dessen Hilfe die programmierspezifischen Implementierungen erzeugt werden. Dies erfolgt allerdings in der Regel zur Entwicklungszeit der Anwendung und nicht zu deren Laufzeit.

Zwar ist es bei Sprach- und Ausführungsumgebungen wie Java heute technisch durchaus möglich, auch zur Ausführungszeit Klassen der Anwendung hinzuzufügen und auf diesem Weg das oben beschriebene Implementierungszenario von der Entwicklungs- in die Laufzeit zu verlagern. Allerdings würde ein solcher Ansatz eine Reihe von Nachteilen bzw. Riksiken bergen. Das gewichtigste Argument gegen den Genierungsansatz ist zweifelsfrei im Bereich Sicherheit angesiedelt. Das Einbinden von nicht getestetem Code bietet geschickten Angreifenr ein \emph{el Dorada} von Möglichkeiten, potentiell gefährliche Programmsequenzen in die Anwendung ein zu schmuggeln.

Statt des generativen Ansatzes eignet sich daher ein Framework, das mit Hilfe der Informationen in einem \ac{WSDL}-Dokument eine \ac{SOAP}-Kommunikation durchführen kann, ohne dazu Codegenerierungen durchführen zu müssen, besser. Für Java-Anwendungen existiert dazu unter anderem das \ac{WSIF} der Apache Group. Entsprechend den Elementen einer WSDL-Beschreibung sind innerhalb des \ac{WSIF} Klassen definiert, die mittels der \ac{WSDL}-Eingabe parametrisiert werden. Damit ist es möglich jeden denkbaren Web Service "`spontan"' zu nutzen und so tatsächlich dynamisches Verhalten der Anwendungen in Web-Service-Szenarien zu erreichen.



In diesem Abschnitt stelle ich zwei konkrete Ansätze zur automatischen, domänenübergreifenden Vermittlung zwischen Webservices vor.

\subsection{SAWSDL}

In \ac{SAWSDL} werden mit einem \emph{liftingSchemaMapping} und einem \emph{loweringSchemaMapping} die Abbildung der Funktionen eines Webservices auf eine Semantik abgebildet.

As we noted in section 1, the technology of the Web
services was accompanied by many standards. However,
these standards do not remedy to the problem of adaptability
to Web services’ changes, and do not cover all aspects related
to different tasks of the Web service’s life cycle, namely, the
discovery, the invocation, the publication and the
composition. Indeed, many enterprises applications, in
particular, can constantly have need to discover and use
existing Web services, or to compose them to meet new
requirements or a complex request (eg. a travel agency can
for example use both fly and hotel booking services to
respond a customer’s query). The number of Web services
which increase on Internet makes their discovery and/or
composition a complex task. Automating these tasks is a
solution which would reduce the cost of the Web services’
implementation. Nevertheless, with this intention, we think
that it is necessary to enrich the description of the Web
services by explicit and comprehensible semantics, which can
be used by the machine. \cite{ei-sawsdl}

The more valuable
gain of SAWSDL lies in opportunities to annotate existing
WSDL descriptions in a bottom-up fashion while at the
same time only use descriptions of services which are relevant
to specific domain requirements. \cite{WSMOLITE}

http://www.docstoc.com/docs/82666005/SAWSDL-tutorial-at-Semantic-Technology-Conference-2007

\subsection{ORISF}

This paper presents the Ontology-based Resourceoriented
Information Supported Framework (ORISF). The
framework’s aim is to support RESTful Web Service with
resource model through ontology evolution and RESTful
service description, and realize the semantic-level
integration of resources. \cite{zg-ontorest}
