Wie im vorigen Abschnitt beschrieben wurde, fehlt der bisherigen Beschreibung von Webservices der semantische Aspekt. Unter \emph{semantischen \ac{WS}} versteht man solche \ac{WS}, deren Beschreibung neben der konkreten technischen Aspekten zur Anbindung auch Information zur abgebildeten "`Welt"' enthalten. In diesem Abschnitt erläutere ich deren Grundlagen.

\subsection{Semantik}

Die \emph{Semantik} (griechisch, "`bezeichnen"') beschreibt das Wesen von Dingen und ermöglicht die Interpretation und Übertragung von Konzepten auf konkrete Begebenheiten. Semantik ist die Grundlage jeglicher Kommunikation und umgibt uns überall. Bereits in jungen Jahren lernen wir, dass ein über einem Weg hängender Kasten, aus dem uns ein Licht rot anstrahlt eine \emph{bestimmte} Bedeutung hat. Nach einiger Zeit verbinden wir damit intuitiv: "`Halt, hier geht es nicht weiter."' Wichtig ist allerdings, dass die scheinbar eindeutige Verbindung zwischen der Farbe "`Rot"' und dem Konzept "`Nicht weiter gehen!"' kontextabhängig ist. Begegnet uns ein leuchtendes Rot auf einem Apfel, wissen wir, dass das Obst frisch und genießbar ist --- die Bedeutung verdreht sich in das Gegenteil.

Wie schon auf Seite \pageref{l:intro-loosecoupling} beschrieben, ist Voraussetzung für eine Service-Infrastruktur mit loser Kopplung, dass die Bedeutung der Aufgabe, die mit dem Webservice abgebildet wird automatisch ermittelt werden kann.

Beschreibt man einen Webservice z.B. mittels der \ac{WSDL}, legt man damit lediglich den Syntax für die vom Webservice verarbeiteten Anfragen fest. Die Bedeutung der Funktionalität und der übertragenen Daten erschließt sich daraus nicht. Sie entsteht lediglich in der Interpretation der Benutzer des Dienstes. 

In Listing \ref{code:wsdl} auf Seite \pageref{code:wsdl} findet sich eine WSDL für der Webservice "`PeopleAsk"'\footnote{http://peopleask.ooz.ie/}, mit dem sich die aktuell in Google gestellten Fragen abrufen lassen. Sie beschreibt die Entitäten \emph{GetQuestionsAbout} mit dem Attribut \emph{query} und \emph{GetQuestionsAboutResponse}, das eine Liste mit Strings ist. Aus dem Dokument geht jedoch nicht hervor, dass eigentlich \emph{Suchanfragen} einer \emph{Suchmaschine} zurückgegeben werden --- dieses Wissen entsteht aus Informationen, die nur außerhalb der Schnittstellenbeschreibung zugänglich sind.

Es fehlt also eine Komponente, die dem reinen Akt der Datenübertragung ein inhaltlichen Beschreibung, hinzufügt und das zudem noch in maschinenlesbarer Form.

\subsection{Ontologien}

In der Informatik sind \emph{Ontologien} die \emph{Spezifikation eines Konzepts}. 

\emph{Spezifikation} bedeutet dabei eine formale und deklarative Repräsentation, die damit automatisch maschinenlesbar ist und Missverständnisse ausschließt. Ein \emph{Konzept} ist die abstrakte und vereinfachte Sicht der für das Konzept relevanten Umgebung.

Ontologien beschreiben aus der Sicht des Dienstanbieters die Zusammenhänge in der Umgebung, auf die durch den Webservice implizit zugegriffen wird.

In \cite{dcswe} liefert Devedžić zum bessern Verständnis dieses Bildnis: Möchte eine Person über Dinge aus der Domäne \emph{D} mit der Sprache \emph{L} sprechen, beschreiben Ontologien die Dinge, von denen angenommen wird, dass sie in \emph{D} existieren als Konzepte, Beziehungen und Eigenschaften von \emph{L}.

\subsection{Semantische Beschreibung}

Mit Hilfe des \ac{RDF}, der \ac{RDFS} und deren Erweiterung \ac{OWL} ist es möglich, Webservices semantisch zu beschreiben. 

In \ac{RDF} werden dabei die Entitäten (in \ac{RDF} "`Ressourcen"' genannt) mit ihren Attributen und Beziehungen untereinander syntaktisch beschrieben \cite{w3c-rdf}.

In \ac{RDF} fehlt aber die Möglichkeit, die Beziehungen von Eigenschaften zu beschreiben. Zum Beispiel besitzt ein Buch das Attribut \emph{Autor}. Dass damit aber eine weitere Ressource gemeint ist (eine \emph{Person} mit der Rolle \emph{Autor}) lässt sich in einem \ac{RDF}-Dokument nicht hinterlegen. Mit \ac{RDFS} wurde deswegen die Möglichkeit geschaffen, Gruppen zusammengehöriger Ressourcen und ihrer Beziehung untereinander zu beschreiben \cite{w3c-rdfs}.

Mit \ac{OWL} ist es schließlich möglich, Ontologien in Form von Klassen, Eigenschaften, Instanzen und Operationen zu beschreiben \cite{w3c-owl2primer}.

Die sich damit bietende Möglichkeit, auch Webservices semantisch zu beschreiben ist also die Voraussetzung dafür, dass Technologien entstehen, mit denen die Vermittlung zwischen zwei Domänen automatisiert statt finden können.

\subsection{SAWSDL}

SAWSDL is the World Wide Web Consortium’s
(W3C; www.w3.org) first step toward standardizing
technologies for Semantic Web services
(SWSs). As a standard, it provides a common
ground for the various ongoing efforts
toward SWS frameworks, such as the
Web Service Modeling Ontology
(WSMO; www.wsmo.org)2 and the
OWL-based Web Service Ontology
(OWL-S; www.daml.org/services/owl-s).\cite[S.60]{ky-sawsdl}

On the semantic side, SAWSDL is independent
of any ontology technology
and assumes that semantic concepts
can be identified via URIs. For instance,
SWS frameworks can use the Resource
Description Framework (RDF) and Web
Ontology Language (OWL) with SAWSDL
to annotate Web services..\cite[S.61]{ky-sawsdl}

SAWSDL is a set of extensions for
WSDL, which provides a standard
description format for Web services.
WSDL uses XML as a common flexible
data-exchange format and applies
XML Schema for data typing. It
describes a Web service on three levels:
\begin{itemize}
\item Reusable abstract interface defines
a set of operations, each representing
a simple exchange of messages
described with XML Schema element
declarations.
\item  Binding describes on-the-wire mes-
sage serialization; it follows the
structure of an interface and fills in
the necessary networking details
(for instance, for SOAP or HTTP).
\item Service represents a single physical
Web service that implements a single
interface; the Web service can
be accessed at multiple network
endpoints.
\end{itemize}

WSDL aims to describe the Web
service on a syntactic level: it specifies
what messages look like rather
than what they mean. SAWSDL is a
simple extension layer on top of
WSDL that lets WSDL components
specify their semantics. SAWSDL
defines extension attributes that we
can apply to elements both in WSDL
and in XML Schema to annotate
WSDL interfaces, operations, and their
input and output messages.
The SAWSDL extensions take two
forms: model references that point to
semantic concepts and schema mappings
that specify data transformations
between messages’ XML data
structure and the associated semantic
model. In Table 1, we summarize
the complete syntax introduced by
SAWSDL.

... bis Seite 63 in \cite{ky-sawsdl}

In \ac{SAWSDL} werden mit einem \emph{liftingSchemaMapping} und einem \emph{loweringSchemaMapping} die Abbildung der Funktionen eines Webservices auf eine Semantik abgebildet.

As we noted in section 1, the technology of the Web
services was accompanied by many standards. However,
these standards do not remedy to the problem of adaptability
to Web services’ changes, and do not cover all aspects related
to different tasks of the Web service’s life cycle, namely, the
discovery, the invocation, the publication and the
composition. Indeed, many enterprises applications, in
particular, can constantly have need to discover and use
existing Web services, or to compose them to meet new
requirements or a complex request (eg. a travel agency can
for example use both fly and hotel booking services to
respond a customer’s query). The number of Web services
which increase on Internet makes their discovery and/or
composition a complex task. Automating these tasks is a
solution which would reduce the cost of the Web services’
implementation. Nevertheless, with this intention, we think
that it is necessary to enrich the description of the Web
services by explicit and comprehensible semantics, which can
be used by the machine. \cite{ei-sawsdl}

The more valuable
gain of SAWSDL lies in opportunities to annotate existing
WSDL descriptions in a bottom-up fashion while at the
same time only use descriptions of services which are relevant
to specific domain requirements. \cite{WSMOLITE}

http://www.docstoc.com/docs/82666005/SAWSDL-tutorial-at-Semantic-Technology-Conference-2007

\begin{itemize}
\item http://www.w3.org/TR/sawsdl-guide/
\item \ac{SAWSDL} und \ac{REST}? \cite{xn-sss}
\item \ac{SAWSDL} verwenden: Grüner Kasten in \cite[S.63]{ky-sawsdl}, \cite{flexbrok} und sehr ausführlich in \cite{vr-sesa}
\end{itemize}