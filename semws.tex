Wie im vorigen Abschnitt beschrieben wurde, fehlt der bisherigen Beschreibung von Webservices der semantische Aspekt. Unter \emph{semantischen \acl{WS}} versteht man solche \acl{WS}, deren Beschreibung neben der konkreten technischen Aspekten zur Anbindung auch Information zur abgebildeten "`Welt"' enthalten. In diesem Abschnitt erläutere ich deren Grundlagen.

\subsection{Semantik}

Die \emph{Semantik} (griechisch, "`Bezeichnung"') beschreibt das Wesen von Dingen und ermöglicht die Interpretation und Übertragung von Konzepten auf konkrete Begebenheiten. Semantik ist die Grundlage jeglicher Kommunikation und umgibt uns überall. Bereits in jungen Jahren lernen wir, dass ein über einem Weg hängender Kasten, aus dem uns ein Licht rot anstrahlt eine \emph{bestimmte} Bedeutung hat. Nach einiger Zeit verbinden wir damit intuitiv: "`Halt, hier geht es nicht weiter."' Wichtig ist allerdings, dass die scheinbar eindeutige Verbindung zwischen der Farbe "`Rot"' und dem Konzept "`Nicht weiter gehen!"' kontextabhängig ist. Begegnet uns ein leuchtendes Rot auf einem Apfel, wissen wir, dass das Obst frisch und genießbar ist. Die Bedeutung "`Halt!"' der Farbe Rot verdreht sich in diesem Kontext in das Gegenteil: "`Iss mich!"'.

Wie schon auf Seite \pageref{l:intro-loosecoupling} beschrieben, ist Voraussetzung für eine Service-Infrastruktur mit loser Kopplung, dass die Bedeutung der Aufgabe, die mit dem Webservice abgebildet wird automatisch ermittelt werden kann.

Beschreibt man einen Webservice z.B. mittels der \ac{WSDL}, legt man damit lediglich den Syntax für die vom Webservice verarbeiteten Anfragen fest. Die Bedeutung der Funktionalität und der übertragenen Daten erschließt sich daraus nicht. Sie entsteht lediglich in der Interpretation der Benutzer des Dienstes. 

In Listing \ref{code:wsdl} auf Seite \pageref{code:wsdl} findet sich eine \ac{WSDL} für der Webservice "`PeopleAsk"'\footnote{http://peopleask.ooz.ie/}, mit dem sich die aktuell in Google gestellten Fragen abrufen lassen. Beschrieben werden die Entitäten \emph{GetQuestionsAbout} mit dem Attribut \emph{query} und \emph{GetQuestionsAboutResponse}, das eine Liste mit Strings ist. Aus dem Dokument geht jedoch nicht hervor, dass eigentlich \emph{Suchanfragen} einer \emph{Suchmaschine} zurückgegeben werden --- dieses Wissen entsteht aus Informationen, die nur außerhalb der Schnittstellenbeschreibung zugänglich sind.

Es fehlt also eine Komponente, die dem reinen Akt der Datenübertragung ein inhaltlichen Beschreibung, hinzufügt und das zudem noch in maschinenlesbarer Form. In der Informatik sind das \emph{Ontologien}.

Ontologien sind die \emph{Spezifikation eines Konzepts}. \emph{Spezifikation} bedeutet dabei eine formale und deklarative Repräsentation, die damit automatisch maschinenlesbar ist und Missverständnisse ausschließt. Ein \emph{Konzept} ist die abstrakte und vereinfachte Sicht der für das Konzept relevanten Umgebung. Ontologien beschreiben aus der Sicht des Dienstanbieters die Zusammenhänge in der Umgebung, auf die durch den Webservice implizit zugegriffen wird. In \cite[S.31]{dcswe} liefert Devedžić zum bessern Verständnis dieses Bildnis: Möchte eine Person über Dinge aus der Domäne \emph{D} mit der Sprache \emph{L} sprechen, beschreiben Ontologien die Dinge, von denen angenommen wird, dass sie in \emph{D} als Konzepte, Beziehungen und Eigenschaften von \emph{L} existieren.

\subsection{Semantische Beschreibung von \acl{WS}}\label{l:sawsdl}

Mit den \ac{SAWSDL} hat das \ac{W3C} 2007 einen Entwurf zu einem Standard vorgelegt, der es ermöglicht Informationen zu diesen Ontologien maschinenlesbar, als Teil einer \ac{WSDL}-Datei, auszuliefern. 

\ac{SAWSDL} ist dabei unabhängig von einem semantischen Konzept und liefert nur den Rahmen um andere semantische Frameworks in \ac{WSDL} zu integrieren --- es wird lediglich vorausgesetzt, dass diese Konzepte anhand von URIs identifierziert werden können \cite[S.61]{ky-sawsdl}. Eine Mögliche Technik zur semantischen Beschreibung ist, \ac{OWL}, das auf \ac{RDF} basiert. In \ac{RDF} werden dabei die Entitäten (in \ac{RDF} "`Ressourcen"' genannt) mit ihren Attributen und Beziehungen untereinander syntaktisch beschrieben \cite{w3c-rdf}. In \ac{RDF} fehlt aber die Möglichkeit, die Beziehungen von Eigenschaften zu beschreiben. Zum Beispiel besitzt ein Buch das Attribut \emph{Autor}. Dass damit aber eine weitere Ressource gemeint ist (eine \emph{Person} mit der Rolle \emph{Autor}) lässt sich in einem \ac{RDF}-Dokument nicht hinterlegen. Mit \ac{RDFS} wurde deswegen die Möglichkeit geschaffen, Gruppen zusammengehöriger Ressourcen und ihrer Beziehung untereinander zu beschreiben \cite{w3c-rdfs}. Mit \ac{OWL} ist es schließlich möglich, Ontologien in Form von Klassen, Eigenschaften, Instanzen und Operationen zu beschreiben \cite{w3c-owl2primer}.

In \ac{WSDL} werden \acl{WS} auf einer syntaktischen Ebene beschrieben. Es wird festgelegt, wie die auszutauschenden Nachrichten \emph{aussehen} und an welchen Endpunkten der \ac{API} diese zum Einsatz kommen, nicht jedoch was sie bedeuten. In \ac{WSDL} werden die abstrakten Elemente \emph{Element Declaration}, \emph{Type Definition} und \emph{Interface} verwendet, um einen Webservice allgemein zu beschreiben, hier spielen die technischen wie das verwendete Protokoll keine Rolle. Ein \emph{Type} entspricht dabei einem Objekt aus der Domäne, ein \emph{Element} beschreibt ein Attribut dieses Objekts. Ein \emph{Interface} beschreibt die Operationen und deren Parameter, die von der Schnittstelle unterstützt werden. In Listing \ref{code:wsdl2} auf Seite \pageref{code:wsdl2} findet sich ein einfaches Beispiel einer \ac{WSDL}-Datei.

\ac{SAWSDL} führt nun für diese drei Elemente zusätzlich Attribute ein, um deren semantische Bedeutung zu beschreiben:

\begin{itemize}
\item \texttt{modelReference} definiert eine Beziehung zwischen einem der definierten Komponente in der \ac{WSDL} und einem Objekt im semantischen Modell. Diese Attribut kann auf jedes \ac{WSDL}- oder XML-Schema-Element angewendet werden. Der Wert des Attributes ist dabei eine oder mehrere URIs, die auf ein semantisches Modell verweisen.
\item Die Attribute \texttt{liftingSchemaMapping} und \texttt{loweringSchemaMapping}, die auf Typedefinitionen definiert werden können, spezifizieren das Mapping zwischen semantischen Daten (z.B. \ac{RDF} und XML) sowie umgekehrt. Hierbei ist es auch möglich, mehrere Mappings je Typ zu definieren, um verschiedene Repräsentation je Kontext zu ermöglichen. Die \emph{lifting}- und \emph{lowering}-Transformationen sind nützlich, wenn von einem semantischen Client aus mit einem \acl{WS} kommunziert wird. Für eine Anfrage werden dann die Semantischen Daten in das Anfrage-Format des Client durch \emph{lowering} transformiert, die Antwort wird dann durch \emph{lifting} wieder in ein semantisches Format konvertiert.
\end{itemize}\cite[S.62ff]{ky-sawsdl}

% TODO: Abbildung 3 aus \cite{ky-sawsdl} rein

Dieses Verfahren kommt auch bei der Verwendung einer gemeinsamen Ontologie zum Einsatz --- ein automatischer Vermittler kann dabei die Daten zwischen zwei Schnittstellen mit den \emph{Lifting}-Informationen des Anfragers und den \emph{Lowering}-Informationen des Empfängers vermitteln.

% TODO: Abbildung 4 aus \cite{ky-sawsdl} rein

% TODO: http://www.docstoc.com/docs/82666005/SAWSDL-tutorial-at-Semantic-Technology-Conference-2007

% MARK

\begin{itemize}
\item http://www.w3.org/TR/sawsdl-guide/
\item \ac{SAWSDL} und \ac{REST}? \cite{xn-sss}
\item \ac{SAWSDL} verwenden: Grüner Kasten in \cite[S.63]{ky-sawsdl}, \cite{flexbrok} und sehr ausführlich in \cite{vr-sesa}
\end{itemize}