
\emph{Semantische Webservices} sind ein Konzept, mit dem es möglich wird, die Anbindung von Webservices abstrakt zu beschreiben und so eine lose Kopplung zu erreichen.

In diesem Abschnitt erläutere ich deren Grundlagen.

Wie schon auf Seite \pageref{l:intro-loosecoupling} beschrieben, ist Voraussetzung für eine Service-Infrastruktur mit loser Kopplung, dass die Bedeutung der Aufgabe, die mit dem Webservice abgebildet wird automatisch ermittelt werden kann.\cite{xmlspek1}

Beschreibt man einen Webservice mittels \ac{WSDL}, legt man damit lediglich den Syntax für die vom Webservice verarbeiteten Anfragen fest, die Bedeutung der Funktionalität und der übertragenen Daten erschließt sich daraus nicht, sondern entsteht lediglich in der Interpretation der Benutzer des Dienstes.

Es fehlt also eine Komponente, die dem reinen Akt der Datenübertragung ein inhaltlichen Beschreibung hinzufügt.

In unserem Beispiel mit dem Intranet und der Zeiterfassung, z.B. die Information, dass Mitarbeiter des Unternehmens eine Entsprechung in \ac{mite} haben, nämlich dort als Benutzer verstanden werden.

\subsection{Semantik}

\emph{Semantik}

\paragraph{Weiterführende Publikationen} In einer Artikelserie aus dem Jahr 2004 (\cite{xmlspek1}, \cite{xmlspek2}, \cite{xmlspek3} und \cite{xmlspek4}) beschreiben Wolfgang Dostal, Mario Jeckle und Werner Kriechbaum ausführlich das Konzept der semantischen Webservices.