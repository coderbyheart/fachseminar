
\emph{Semantische Webservices} sind ein Konzept, mit dem es möglich wird, die Anbindung von Webservices abstrakt zu beschreiben und so eine lose Kopplung zu erreichen. In diesem Abschnitt erläutere ich deren Grundlagen.

\subsection{Semantik}

Wie schon auf Seite \pageref{l:intro-loosecoupling} beschrieben, ist Voraussetzung für eine Service-Infrastruktur mit loser Kopplung, dass die Bedeutung der Aufgabe, die mit dem Webservice abgebildet wird automatisch ermittelt werden kann.\cite{xmlspek1}

Beschreibt man einen Webservice z.B. mittels der \ac{WSDL}, legt man damit lediglich den Syntax für die vom Webservice verarbeiteten Anfragen fest, die Bedeutung der Funktionalität und der übertragenen Daten erschließt sich daraus nicht, sondern entsteht lediglich in der Interpretation der Benutzer des Dienstes. In Listing \ref{code:wsdl} findet sich eine WSDL für einen Webservice, mit dem sich die aktuell in Google gestellten Fragen abrufen lassen\footnote{http://peopleask.ooz.ie/}. In der Beschreibung finden sich aber lediglich die Entitäten \emph{GetQuestionsAbout} mit dem Attribut \emph{query} und \emph{GetQuestionsAboutResponse} \emph{GetQuestionsAbout}, das eine Liste mit Strings ist. Aus dem Dokument geht nicht hervor, dass eigentlich \emph{Suchanfragen} einer \emph{Suchmaschine} zurückgegeben werden --- dieses Wissen ensteht aus Informationen, die außerhalb der Schnittstellenbeschreibung zugänglich sind.

Auf unsere Beispiel mit dem Intranet und der Zeiterfassung übertragen, fehlt z.B. die Information, dass Mitarbeiter des Unternehmens eine Entsprechung in \ac{mite} haben, nämlich dort als Benutzer verstanden werden.

Es fehlt also eine Komponente, die dem reinen Akt der Datenübertragung ein inhaltlichen Beschreibung, hinzufügt und das zudem noch in maschinenlesbarer Form. \cite{xmlspek2}

Die \emph{Semantik} (griechisch, "`bezeichnen"') beschreibt das Wesen von Dingen und ermöglicht die Interpretation und Übertragung von Konzepten auf konkrete Begebenheiten. Semantik ist die Grundlage jeglicher Kommunikation und umgibt uns überall. Bereits in jungen Jahren lernen wir, dass ein über einem Weg hängender Kasten, aus dem uns ein Licht rot anstrahlt eine \emph{bestimmte} Bedeutung hat. Intuitiv verbinden wir damit: "`Halt, hier geht es nicht weiter."' Wichtig ist allerdings, dass die scheinbar eindeutige Verbindung zwischen der Farbe "`Rot"' und dem Konzept "`Nicht weiter gehen!"' kontextabhängig ist. Begegnet uns ein leuchtendes Rot auf einem Apfel, wissen wir, dass das Obst frisch und genießbar ist --- die Bedeutung verdreht sich in das Gegenteil.

\paragraph{Weiterführende Publikationen} In einer Artikelserie aus dem Jahr 2004 (\cite{xmlspek1}, \cite{xmlspek2}, \cite{xmlspek3} und \cite{xmlspek4}) beschreiben Wolfgang Dostal, Mario Jeckle und Werner Kriechbaum ausführlich das Konzept der semantischen Webservices.