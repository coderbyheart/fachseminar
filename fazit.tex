In dieser Seminararbeit wurde gezeigt, dass es möglich ist, Architekturen zu entwickeln, in denen Services dynamisch gebunden werden können. Damit das möglich ist muss zu diesen Services neben der syntaktischen Beschreibung auch eine semantische Beschreibung vorliegen. Erst so ist es möglich, aus einer Menge von Diensten unabhängig von technischen Parametern einen passenden auszuwählen. Die vorgestellten Lösungen haben aber immer noch einen Nachteil: die semantischen Konzepte werden an einer zentralen Stelle beim Betreiber der Architektur verwaltet und beziehen sich auch auf ein gemeinsames Verständnis der Domäne. Es ist also auch weiterhin erforderlich, dass sich Dienstanbieter und Dienstkonsumenten auf einen gemeinsamen Nenner einigen. 
Angewendet auf den Bereich der \emph{komplexen Webanwendungen} muss dies aber kein Nachteil sein. Statt einer "`festverdrahteten"' Implementierung muss zur Integration eines neuen Services lediglich eine semantische Beschreibung entworfen werden, die die vom Service zur Verfügung gestellten Daten in Ontologien des Unternehmens ausdrückt, was ohne Änderungen am Quellcode einer \ac{SOA} möglich. Es existieren auch Bestrebungen, allgemeine Ontologien\footnote{Ein Liste findet sich z.B. unter \url{http://semanticweb.org/wiki/Ontology}} zu definieren, so dass es in Zukunft immer leichter möglich sein wird, Gemeinsamkeiten in verschiedenen Domänen zu finden.